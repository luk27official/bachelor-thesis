

%%% Choose a language %%%

\newif\ifEN
\ENtrue     % uncomment this for english
%\ENfalse   % uncomment this for czech

%%% Configuration of the title page %%%

\def\ThesisTitleStyle{mff} % MFF style
%\def\ThesisTitleStyle{cuni} % uncomment for old-style with cuni.cz logo
%\def\ThesisTitleStyle{natur} % uncomment for nature faculty logo

\def\UKFaculty{Faculty of Mathematics and Physics}
%\def\UKFaculty{Faculty of Science}

\def\UKName{Charles University in Prague} % this is not used in the "mff" style

% Thesis type names, as used in several places in the title
\def\ThesisTypeTitle{\ifEN BACHELOR THESIS \else BAKALÁŘSKÁ PRÁCE \fi}
%\def\ThesisTypeTitle{\ifEN MASTER THESIS \else DIPLOMOVÁ PRÁCE \fi}
%\def\ThesisTypeTitle{\ifEN RIGOROUS THESIS \else RIGORÓZNÍ PRÁCE \fi}
%\def\ThesisTypeTitle{\ifEN DOCTORAL THESIS \else DISERTAČNÍ PRÁCE \fi}
\def\ThesisGenitive{\ifEN bachelor \else bakalářské \fi}
%\def\ThesisGenitive{\ifEN master \else diplomové \fi}
%\def\ThesisGenitive{\ifEN rigorous \else rigorózní \fi}
%\def\ThesisGenitive{\ifEN doctoral \else disertační \fi}
\def\ThesisAccusative{\ifEN bachelor \else bakalářskou \fi}
%\def\ThesisAccusative{\ifEN master \else diplomovou \fi}
%\def\ThesisAccusative{\ifEN rigorous \else rigorózní \fi}
%\def\ThesisAccusative{\ifEN doctoral \else disertační \fi}



%%% Fill in your details %%%

\def\ThesisTitle{{Extension of web-based interface for protein binding sites prediction}}
\def\ThesisAuthor{{Lukáš Polák}}
\def\YearSubmitted{{2023}}

% department assigned to the thesis
\def\Department{{Department of Software Engineering}}
% Is it a department (katedra), or an institute (ústav)?
\def\DeptType{{Department}}

\def\Supervisor{{doc. RNDr. David Hoksza, Ph.D.}}
\def\SupervisorsDepartment{{Department of Software Engineering}}

% Study programme and specialization
\def\StudyProgramme{{Programming and software development Bc.}}
\def\StudyBranch{{IPP2}}

\def\Dedication{%
Dedication. \xxx{It is nice to say thanks to supervisors, friends, family, book authors and food providers.}
}

\def\AbstractEN{%
Protein-ligand binding sites are positions on the protein structure where the protein interacts with other molecules. PrankWeb is a web server developed at MFF UK allowing prediction of such places. These predictions are essential in fields such as bioengineering and computational drug discovery. The goal of this thesis was to update this web server, i.e., replace old and unsupported components with new ones. Another goal was to extend the server architecture to enable the simple addition of modules for the postprocessing of the predicted binding sites. These modules can be implemented either on the client side in case of simple computations, or on the server side in case of complex computations. As part of the thesis, we have implemented a client module for computing the volume of active sites and a server module allowing the docking of small proteins into predicted binding sites. The thesis describes not only the interventions in the architecture but also provides a short introduction to the problem of protein-ligand binding sites and their prediction.
}

\def\AbstractCS{%
Protein-ligand vazebná místa jsou pozice na struktuře proteinu, kde protein interaguje s dalšími molekulami. PrankWeb je webový server vyvinutý na MFF UK umožňující predikci takových míst. Tyto predikce jsou zásadní pro bioengineering nebo počítačový vývoj léčiv. Cílem práce bylo aktualizovat tento webový server, tedy nahradit staré a nepodporované komponenty za nové. Dalším cílem bylo rozšířit architekturu serveru pro snadné přidávání dalších modulů sloužících pro postprocessing predikovaných vazebných míst. Tyto moduly mohou být implementovány buď na frontendu v případě jednoduchých výpočtů, nebo na backendu v případě výpočtů komplexních. V rámci práce jsme implementovali klientský modul pro výpočet objemu aktivních míst a serverový modul umožňující dockování malých proteinů do vazebných predikovaných míst. Práce popisuje nejen zásahy do architektury, ale obsahuje i stručný vhled do problematiky protein-ligand vazebných míst a jejich predikce.
}

% 3 to 5 keywords (recommended), each enclosed in curly braces.
% Keywords are useful for indexing and searching for the theses by topic.
\def\Keywords{%
{{bioinformatics} {web} {software} {protein}}
}

% If your abstracts are long and do not fit in the infopage, you can make the
% fonts a bit smaller by this setting. (Also, you should try to compress your abstract more.)
% Alternatively, consider increasing the size of the page by uncommenting the
% geometry modification in thesis.tex.
\def\InfoPageFont{}
%\def\InfoPageFont{\small}  %uncomment to decrease font size

\ifEN\relax\else
% If you are writing a czech thesis, you additionally need to fill in the
% english translation of the metadata here!
\def\ThesisTitleEN{{Extension of web-based interface for protein binding sites prediction}}
\def\DepartmentEN{{Department of Software Engineering}}
\def\DeptTypeEN{{Department}}
\def\SupervisorsDepartmentEN{{Department of Software Engineering}}
\def\StudyProgrammeEN{{Programming and software development Bc.}}
\def\StudyBranchEN{{IPP2}}
\def\KeywordsEN{%
{{bioinformatics} {web} {software} {protein}}
}
\fi
