
\chapwithtoc{Introduction}

Detection of protein ligand-binding sites is a vital aspect of nowaday's drug discovery and development. Identification of the potential binding sites allows understanding of various molecule interactions, which is the first step of rational drug discovery pipelines. Recognizing the binding sites is thus very important for further bioinformatic research and studies \cite{10.1093/bioinformatics/btt447}.

P2Rank is a machine-learning based tool developed at MFF UK which allows users to predict the ligand-binding sites for a given protein structure. P2Rank works standalone and outperforms most of the existing binding sites prediction tools \cite{krivak2018p2rank}. PrankWeb is a web-based tool that provides an user-friendly interface for P2Rank. A significant difference between PrankWeb and other web-based tools is that PrankWeb does not employ either JMol nor JSMol for structure visualization of the results \cite{jendele2019prankweb}. These libraries are rather old-fashioned. The original PrankWeb auhtors decided to use the LiteMol for structure visualization and Protael for sequence visualization. Although these libraries are still working, they are not actively developed anymore.

There were two main goals of this thesis. The first goal was to update the PrankWeb interface to use different libraries for the visualization, namely MolStar \cite{10.1093/nar/gkab314} and RCSB Saguaro 1D Feature Viewer \cite{10.1093/bioinformatics/btaa1012}. This improves not only the visual appearance of the results, but also the performance. Furthermore, the updated libraries are actively developed and thus potentially more reliable in the future. The second goal was to update the PrankWeb architecture, so that post-processing may be done on the predicted binding sites both on the client-side and the server-side. This introduces a potential to create custom plug-ins for the web interface.

The thesis presents a working version of the updated PrankWeb tool. The updated tool should keep the current functionality of the website and visualize the results via a more user-friendly interface. The tool should also be able to perform computations on the predicted binding sites.

In the first chapter, the reader will be introduced to the basics of protein and ligand-binding sites problematic, later in this chapter the current PrankWeb architecture and the P2Rank tool itself will be covered. The reader will also be briefly introduced to similar web-tools. The second chapter will cover the programming part of the work. Firstly, the usage of the updated libraries and other frontend design decisions will be introduced. Secondly, the plug-in architecture including both the client-side and the server-side computations will be covered in the respective subchapters. In the third chapter, the tool usage will be described from two perspectives - a user and a developer. The conclusion will cover the results and the potential extensions to the tool.
