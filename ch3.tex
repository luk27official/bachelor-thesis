\chapter{User documentation}
\label{chap:user_docs}

\xxx{TODO: add motivation for users, distinguish between a regular user and a developer}

\section{Deployment}
\label{sec:deployment}

This section describes how to deploy the application. The preffered way is to deploy PrankWeb using Docker. Local development and deployment is limited.

\subsection{Docker deployment}
\label{subsec:docker_deployment}

Firstly, Docker and Docker Compose need to be installed\footnote{Downloadable from the official website at \url{https://docs.docker.com/get-docker/}}. 

After installing Docker, the PrankWeb repository needs to be cloned, preferrably using Git\footnote{Downloadable from the official website at \url{https://git-scm.com/downloads}}.

\begin{enumerate}
    \item Create a directory for the PrankWeb repository and navigate to it.
    \item Clone the repository using the following command:
    \begin{lstlisting}
        git clone https://github.com/cusbg/prankweb.git .
    \end{lstlisting}
    \item Create directories for each of the volumes.
    \item Create mounts for each of the volumes in the \texttt{docker-compose.yml} file. The mounts are created as follows, simply update the \texttt{/tmp/} paths to the paths of the directories created in the previous step:
    \begin{lstlisting}
        docker volume create --name prankweb_rabbitmq --opt type=none --opt device=/tmp/rabbitmq --opt o=bind

        docker volume create --name prankweb_conservation --opt type=none --opt device=/tmp/conservation --opt o=bind

        docker volume create --name prankweb_predictions --opt type=none --opt device=/tmp/predictions --opt o=bind

        docker volume create --name prankweb_services --opt type=none --opt device=/tmp/services --opt o=bind
    \end{lstlisting}
    \xxx{TODO: maybe add docking}
    \item Optionally, create a \texttt{.env} file to change the default variables defined in the \texttt{docker-compose.yml} file. Notice that UID and GID need to have write permissions to the directories created in the previous steps.
    \item Build the Docker images using the following command:
    \begin{lstlisting}
        docker-compose build
    \end{lstlisting}
    \item Optionally, download the conservation database (keep in mind that this database is large, around 30 GB):
    \begin{lstlisting}
        docker-compose run --rm executor python3 /opt/hmm-based-conservation/download_database.py
    \end{lstlisting}
    \item Start the containers using the following command:
    \begin{lstlisting}
        docker-compose up
    \end{lstlisting}
\end{enumerate}

The application is now be accessible at \url{http://localhost:8020/}.

Although the main deployment process will not change much, the best way to get the current information is to refer to the official documentation at \url{https://github.com/cusbg/p2rank-framework/wiki/PrankWeb-deploy-with-Docker}.

\subsection{Local deployment}
\label{subsec:local_deployment}

\xxx{TODO: how to deploy via Docker}

\section{Regular user}
\label{sec:regular_user}
\xxx{TODO: add some screenshots and a brief description of the UI}

\section{Developer}
\label{sec:developer}
\xxx{TODO: add a brief description on how to add new client/server side tasks, change the current visuals etc.}
