
\chapwithtoc{Conclusion}
\label{chap:conclusion}

There were two main goals of this thesis. The first one was to update the existing PrankWeb frontend with newer technologies and to improve the visual design of the application. This was done by introducing the 1D RCSB Saguaro Viewer and Mol* libraries for the structure visualization and by using React to provide a good-looking and responsive user interface. The second goal was to introduce a possibility to add new plug-ins to PrankWeb that allow further postprocessing of the pockets. This was done by creating two interfaces for client-side and server-side plug-ins.

Both of the goals were successfully achieved. The new frontend is more user-friendly and thanks to the used libraries, the visualization is faster than ever before. The plug-in system allows the developers to easily add new features to PrankWeb and deploy these features to their specific users.

Although the current state of PrankWeb is good, there is still room for improvement. There are multiple possibilities to improve the application, on the frontend, the results of docking could be visualized in the Mol* viewer, the 1D viewer could be improved by resizing dynamically according to the window size, and the options for the user could be extended. On the backend, we could add a better program for the docking plug-in, add more plug-ins, and improve the performance of the application.

Still, we believe that the current state of PrankWeb has improved and the application ready to be used by the scientific community.

\xxx{TODO: fix passive voice throughout the thesis (run through Grammarly)}

\xxx{TODO: normalize the citation formatting throughout the thesis}
